\documentclass[main.tex]{subfiles}

\begin{document}

\section{Dataset Description}\label{sec:dataset}

\subsection{Dataset Selection}
% Describe the dataset chosen from UCI ML Repository or Kaggle
% Explain why this dataset was selected
% Mention the number of instances, features, and classes

\subsection{Data Preprocessing}
% Describe the preprocessing steps performed

\subsubsection{Handling Missing Data}
% Explain how missing or noisy data was handled

\subsubsection{Feature Normalization}
% Describe normalization or standardization techniques applied

\subsubsection{Data Splitting}
% Explain how data was split into training, validation, and test sets
% Mention the ratios used (e.g., 60-20-20 or 70-15-15)

\subsubsection{Feature Engineering (Optional)}
% If applicable, describe any dimensionality reduction (PCA) or feature selection performed


\section{Methodology}\label{sec:methodology}

\subsection{Models Implemented}
% Brief overview of all models used in the study

\subsubsection{Naïve Bayes}
% Describe the Naïve Bayes classifier (Gaussian or Multinomial)
% Explain the assumptions and theoretical background

\subsubsection{Logistic Regression}
% Describe Logistic Regression for binary classification
% Explain the loss function and optimization method

\subsubsection{Softmax Regression (Optional)}
% If multi-class: Describe Softmax Regression
% Explain how it extends logistic regression to multiple classes

\subsubsection{Decision Tree}
% Describe the Decision Tree algorithm
% Explain splitting criteria (Gini, entropy)

\subsubsection{Random Forest}
% Describe the Random Forest ensemble method
% Explain how it combines multiple decision trees

\subsubsection{Support Vector Machine}
% Describe SVM with linear and kernel-based approaches
% Explain the margin maximization concept

\subsection{Hyperparameter Tuning}
% Describe the cross-validation strategy used
% Explain which hyperparameters were tuned for each model
% Mention the search method (grid search, random search, etc.)


\section{Results}\label{sec:results}

\subsection{Model Performance}
% Present the performance metrics for each model
% Include tables with Accuracy, Precision, Recall, F1 Score

\subsection{Confusion Matrices}
% Display confusion matrices for each classifier

\subsection{ROC Curves and AUC}
% Present ROC curves and AUC scores
% Compare different models

\subsection{Training vs. Validation Performance}
% Show learning curves
% Discuss overfitting/underfitting observations

\subsection{Computational Cost (Optional)}
% Compare training times and computational requirements


\section{Comparative Analysis}\label{sec:analysis}

\subsection{Best Performing Models}
% Discuss which models performed best and why
% Relate performance to dataset characteristics

\subsection{Model Assumptions and Performance}
% Analyze how model assumptions influence performance
% Discuss cases where certain models excel or fail

\subsection{Overfitting Trade-off}
% Observations on bias-variance trade-off
% Discuss regularization effects

\subsection{Visualizations}
% Include visualizations such as:
% - Learning curves
% - Decision boundaries (if feasible)
% - Feature importance (for tree-based models)

\subsubsection{Learning Curves}
% Show and discuss learning curves

\subsubsection{Decision Boundaries}
% If applicable, show 2D decision boundaries

\subsubsection{Feature Importance}
% For Random Forest and Decision Trees, show feature importance

\end{document}